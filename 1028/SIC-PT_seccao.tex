
Secção A – Agricultura, produção animal, caça, floresta e pesca

        • A CAE-Rev.3 passou a incluir na mesma Secção a agricultura e a pesca, ao contrário da CAE-
          Rev.2.1 em que constituem Secções independentes;

        • A actividade agrícola compreende a produção agrícola e animal, quer em termos de bens, quer de
          serviços específicos das actividades desta Secção;

        • As unidades agrícolas de produção mista classificam-se de acordo com a sua actividade principal,
          enquanto que para as unidades de exploração agrícola e animal em regime de associação é
          necessário determinar previamente um rácio de especialização. As cooperativas agrícolas são
          classificadas em função da sua actividade principal;

        • A Pesca compreende, para além da actividade da pesca, a apanha de algas e de outros produtos
          de águas marítimas e interiores e a aquicultura de espécies piscícolas e afins em regime controlado;

        • As unidades prestadoras de serviços às actividades da pesca classificam-se nas Subclasses
          donde decorre a produção física dos bens;
28


             • As unidades produtoras de vinho ou outro produto agrícola transformado a partir de actividades
               agrícolas a montante (ex: cultura da vinha) são classificadas, regra geral, na Agricultura.

     Secção B - Indústrias Extractivas

             • Além da extracção dos produtos em natureza (sólidos, líquidos e gasosos), esta Secção
               compreende alguma beneficiação feita no local da extracção;

             • A refinação do sal, a aglomeração de carvões e de minérios, associadas ou independentes da
               extracção, passaram para o âmbito da indústria transformadora na CAE-Rev.3;

             • Esta Secção, apesar de não manter as duas Subsecções (extracção de produtos energéticos e
               extracção de produtos não-energéticos), permite assegurar a nível da Divisão o mesmo tratamento
               estatístico para a área da energia.

     Secção C - Indústrias Transformadoras

             • Esta Secção deixou de incluir o nível Subsecção, apresentado as novas Divisões criadas ou
               reorganizadas (ex: 31 – Fabricação de mobiliário e colchões) melhorias acrescidas em termos de
               homogeneidade;

             • A maioria das Divisões da Secção D (Indústrias transformadoras) da CAE-Rev.2.1 permanece no
               âmbito da Secção C (Indústrias transformadoras) da CAE-Rev.3, excepto as Divisões 22 (Edição,
               impressão e reprodução de suportes de informação gravados) e 37 (Reciclagem) em que parte
               significativa do seu âmbito passou, respectivamente, para a Secção J e Secção E;

             • A reparação e a instalação de máquinas e de equipamento, que na CAE-Rev.2.1 pertenciam ao
               âmbito das actividades de fabricação do respectivo equipamento, constituem na CAE-Rev.3 a
               Divisão 33 (Reparação, manutenção e instalação de máquinas e equipamentos);

             • A reconstrução e a conversão de embarcações, aeronaves e de material circulante para caminhos-
               de-ferro classificam-se nas Subclasses que os produzem;

             • As indústrias transformadoras produzem bens e serviços. Os serviços industriais importantes e
               executados por conta de terceiros, encontram-se individualizados em actividades.

     Secção D – Electricidade, gás, vapor, água quente e fria e ar frio

             • Esta Secção apresenta-se como uma parte importante da área energética, encontrando-se as
               partes restantes na Secção B (extracção do carvão, petróleo, urânio e gás) e Secção C (fabricação
               de coque, produtos petrolíferos refinados, combustível nuclear e aglomerados combustíveis);

             • Compreende, além da produção e distribuição de electricidade e gás, alguns serviços específicos
               (ex: comércio de electricidade, comércio de gás por condutas) e a produção de gelo, de vapor de
               água quente;

     Secção E – Captação, tratamento e distribuição de água; saneamento gestão de resíduos e despoluição

             • Esta Secção resulta da agregação de partes de Secções da CAE-Rev.2.1, em particular da Divisão
               37 da Secção D, Divisão 41 da Secção E e Divisão 90 da Secção O;

             • Alem da captação, tratamento e distribuição de água, compreende a recolha, tratamento, eliminação,
               desmantelamento, descontaminação e valorização de resíduos.

     Secção F - Construção

             • A actividade de construção engloba a construção propriamente dita e a demolição (“desconstrução”),
               no âmbito da construção de edifícios e da engenharia civil, sendo as obras o resultado de
               actividades diversas;
                                                                                                                  29


       • Nem todas as actividades que concorrem para a edificação de tais obras estão compreendidas no
         âmbito desta Secção (ex: fabricação de materiais de construção, montagem ou instalação de
         equipamentos industrias que se classificam na Secção C). A montagem ou instalação de
         equipamentos concebidos para que um edifício funcione como tal (ex: instalação eléctrica) pertence
         ao âmbito da Construção;

       • Esta Secção inclui a promoção imobiliária que na CAE-Rev.2.1 pertencia a uma outra Secção
         (Secção K).

Secção G – Comércio por grosso e a retalho; reparação de veículos automóveis e motociclos

       • Esta Secção engloba todas as formas de comércio e a reparação de veículos automóveis e
         motociclos. As Divisões desta Secção compreendem o comércio, reparação e manutenção de
         veículos automóveis e motociclos (Divisão 45), o comércio por grosso e seus agentes (Divisão
         46) e o comércio a retalho (Divisão 47);

       • Os agentes do comércio por grosso têm Subclasses específicas na Divisão 46 para a sua
         classificação, enquanto os agentes do comércio a retalho não efectuado em estabelecimentos
         são classificados na Subclasse 47990;

       • Certas categorias de produtos, pela sua especificidade em termos de comércio, não são
         classificados no comércio a retalho, não existindo por tal facto um paralelismo entre as Divisões
         46 e 47;

       • No comércio a retalho, os Grupos 478 e 479 tratam do comércio não efectuado em estabelecimentos
         (correspondência, Internet, bancas, feiras, distribuição automática, etc.) e os grupos 471, 472,
         473, 474, 475, 476 e 477 correspondem ao comércio a retalho efectuado em estabelecimentos. O




                                                                                                                  CAE - REV. 3
         Grupo 471 respeita ao comércio não-especializado (de predominância alimentar ou não) e os
         Grupos 472, 473, 474, 475, 476 e 477 referem-se ao comércio especializado (alimentar ou não);

       • A reparação de bens de uso pessoal e doméstico foi transferida da Secção do comércio na CAE-
         Rev.2.1 para a Secção S (Outras actividades de serviços) da CAE-Rev.3.

Secção H – Transportes e armazenagem

       • O transporte pode resultar de uma prestação colectiva ou individualizada (ex: táxi), assim como o
         aluguer com condutor de um meio de transporte;

       • Esta Secção inclui, para além do transporte propriamente dito, um conjunto vasto de actividades
         mais ou menos associadas ao transporte (armazenagem, manuseamento de carga, gestão de
         infraestruturas de transportes, organização do transporte, etc.), as actividades postais e de courier;

       • As actividades das telecomunicações e das agências de viagem que na CAE-Rev.2.1 estavam
         ligadas à Secção dos transportes passaram para o âmbito, respectivamente, da Secção J e da
         Secção N, na CAE-Rev.3.

Secção I – Alojamento, restauração e similares

       • O alojamento classificado nesta Secção corresponde ao alojamento de curta duração e engloba,
         quer as unidades hoteleiras, quer outros locais de curta duração;

       • A restauração (restaurantes e similares) compreende os restaurantes propriamente ditos, casas
         de pasto, estabelecimentos de bebidas e similares em que a alimentação e as bebidas são
         consumidas, regra geral, no próprio local, assim como cantinas e fornecimentos de refeições ao
         domicílio (catering);
30



     Secção J - Actividades de informação e de comunicação

             • Esta Secção resultou da agregação de Divisões de várias Secções da CAE-Rev.2.1, em particular,
               Divisão 22 (Edição, impressão e reprodução de suportes de informação gravados) da Secção D,
               Divisão 64 (Telecomunicações) da Secção I, Divisão 72 (Actividades informáticas e conexas) da
               Secção K, 74 (outras actividades de serviços prestados principalmente às empresas) da Secção
               K e 92 (Actividades recreativas, culturais e desportivas) da Secção O;

             • Esta nova Secção (Secção J) permite uma melhor organização da informação estatística para um
               conjunto de actividades quer pela sua homogeneidade, quer pela sua relevância económica.

     Secção K - Actividades financeiras e de seguros

             • As actividades financeiras incluem as unidades de intermediação monetária (banca em sentido
               geral), as unidades de intermediação financeira (actividades financeiras realizadas por entidades
               diferentes das instituições monetárias), seguros, fundos de pensões e actividades auxiliares de
               intermediação financeira, de seguros e de fundos de pensões;

             • De salientar em relação à CAE-Rev.2.1 a criação das Classes, decorrentes da NACE-Rev.2, 6420
               (Actividades das sociedades gestoras de participações sociais) e 6430 (Truts, fundos e entidades
               financeiras similares).

     Secção L - Actividades Imobiliárias

             • Esta Secção passou a incluir só as actividades imobiliárias (ex: compra, venda, arrendamento,
               administração e mediação imobiliária);

             • A promoção imobiliária passou para o âmbito de Secção F (Construção);

             • A limitação do âmbito desta Secção às actividades imobiliárias decorre da sua importância para
               efeitos de Contas Nacionais.

     Secção M – Actividades de consultoria, científicas, técnicas e similares

             • Esta Secção resultou da agregação de Divisões da Secção K da CAE-Rev.2.1, em particular das
               Divisões 73 (Investigação e desenvolvimento) e 74 (Outras actividades de serviços prestados
               principalmente às empresas), cobrindo um conjunto de actividades com um elevado nível de
               especialização e de conhecimentos;

             • As actividades veterinárias (Divisão 85 na CAE-Rev.2.1) passaram a integrar esta Secção;

             • Esta Secção permite uma melhor organização da informação estatística para um conjunto de
               actividades de elevada importância económica e alto grau de homogeneidade.

     Secção N – Actividades administrativas e dos serviços de apoio

             • Esta Secção resultou da agregação da Classe 6330 (Agências de viagens e de turismo e de outras
               actividades de apoio turístico) da Secção I e de Divisões da Secção K da CAE-Rev.2.1, em particular
               das Divisões 71 (Aluguer de máquinas e de equipamentos sem pessoal e bens pessoais e
               domésticos) e 74 (Outras actividades de serviços prestados principalmente às empresas),
               englobando um conjunto de actividades de apoio geral às operações das empresas e que incidem
               sobre a transferência de conhecimento especializado;

             • Esta Secção permite uma melhor organização da informação estatística para as actividades aqui
               incluídas e que têm um elevado grau de homogeneidade entre si.
                                                                                                              31



Secção O - Administração Pública e Defesa; Segurança Social Obrigatória

        • O conceito de Administração Pública é entendido como o conjunto de actividades de
          regulamentação e apoio à gestão de actividades que, pela sua natureza, não podem exercer-se
          numa base de mercado;

        • O estatuto jurídico ou institucional não é determinante para classificar nesta Secção as unidades
          do “tipo administrativo”. Há actividades (ex: ensino, saúde) que não pertencem ao âmbito desta
          Secção, ainda que a Administração Pública desenvolva estas actividades num nível mais ou
          menos elevado;

Secção P - Educação

        • Esta Secção compreende, para além do ensino a todos os níveis e formas, as actividades dos
          institutos e das academias militares, escolas de condução, formação profissional e de ensino
          artístico;

        • Esta Secção passou a incluir os serviços de apoio às actividades educativas.

Secção Q – Actividades de saúde humana e apoio social

        • As actividades dirigidas à saúde humana (hospitalares, liberais, paramédicas, etc.), exercidas em
          regime de internamento ou ambulatório, com ou sem fim lucrativo, estão definidas nesta Secção;

        • No âmbito do apoio social estão incluídas as actividades dos serviços dos equipamentos sociais,
          públicos ou privados, com ou sem alojamento;




                                                                                                              CAE - REV. 3
        • As actividades veterinárias deixaram de incluir esta Secção, passando para o âmbito da Secção M
          (Actividades de consultoria, científicas, técnicas e similares).

Secção R – Actividades artísticas, de espectáculos, desportivas e recreativas

        • Esta Secção inclui actividades culturais, recreativas, desportivas e artísticas;

        • Esta Secção apresenta-se mais homogénea do que a Secção da CAE-Rev.2.1 onde estavam
          compreendidas estas actividades.

Secção S – Outras Actividades de serviços

        • Esta Secção inclui actividades associativas e a reparação de bens de uso pessoal e doméstico;

        • Compreende as actividades dos serviços pessoais não incluídos noutras Secções.

Secção T – Actividades das famílias empregadoras de pessoal doméstico e actividades de produção
das famílias para uso próprio

        • Compreende as actividades dos empregados domésticos enquanto trabalhadores das famílias e
          produção de bens e serviços para uso próprio das famílias.

Secção U – Actividades dos organismos internacionais e outras instituições extra-territoriais


        • Esta secção inclui as actividades das organizações internacionais, embaixadas, consulados e de
          outras instituições extraterritoriais, com imunidade diplomática, estabelecidas em Portugal.
